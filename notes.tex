\documentclass[12pt]{article}

\usepackage[french]{babel} % Document en français
\usepackage[T1]{fontenc} % Suppression d'un warning
\usepackage[utf8]{inputenc} % Document UTF8
\usepackage{graphicx} % Insertion d'images
\usepackage[left=2.2cm, right=2.2cm, top=2.5cm, bottom=2.5cm]{geometry} % Mise en page
\usepackage{multicol} % Texte en multicolonnes avec multicols
\usepackage{placeins}

\graphicspath{{res/}}

\title{LECGE1321 - Management Humain}
\author{Florian Thuin}


\begin{document}

\maketitle
\tableofcontents

\section{Introduction}
  \subsection{Le champ du management humain}
  \subsection{Cas type : France Telecom}
  \subsection{Un outil d'analyse : les niveaux d'Ardoino}
  Les niveaux d'intelligibilité d'Ardoino sont un modèle d'analyse d'une réalité sociale. Celui-ci permet de mettre en défaut la pensée simpliste qui consiste à attribuer une erreur à un facteur individuel (\og{} erreur fondamentale d'attribution\fg{}). La théorie fondamentale de l'attribution consiste d'une part à attribuer sa propre réussite à soi-même et ses échecs à un contexte et d'autre part à attribuer la réussite des autres à un contexte et leurs échecs à eux-mêmes.
  
  Il y a 5 niveaux :
  
  \begin{enumerate}
   \item Le niveau individuel
   \item Le niveau relationnel et groupal
   \item Le niveau organisationnel
   \item Le niveau institutionnel
   \item Le niveau d'historicité
  \end{enumerate}
  
  % TODO : Include l'image du modèle d'Ardoino
  
  \begin{description}
   \item[Niveau individuel] : les facteurs individuels d'ordre psychologiques (attitude, caractère, personnalité, motivations,...)
   \item[Niveau relationnel et groupal] : les relations interpersonnelles et affectives (la communication). Ce niveau étudie les phénomènes de groupe : cohésion, appartenance, esprit de groupe, amitié, complicité, conflits interpersonnels.
   \item[Niveau organisationnel] : les modalités d'organisation de l'action collective. Ce niveau étudie les problèmes d'efficacité : les règles, lee rôles, les status, les modes de division et de coordination du travail, la structure du pouvoir,... 
   \item[Niveau institutionnel] : les valeurs, les normes, la culture. Ce niveau étudie les institutions publiques, les cadres politique, juridique, social et économique ainsi que les règles formelles (lois) et informelles (culture, traditions) qui régissent l'ensemble de la société.
   \item[Niveau d'historicité] : analyse des autres dimensions à travers l'histoire. Ce niveau étudie les transformations de la société, les mouvements sociaux, les tendances économiques, les rapports de force entre les classes sociales,...
  \end{description}
  \subsection{Management humain : de quoi parle-t-on ?}
    \subsubsection{People management}
      Le people management correspond à l'encadrement des employés et au leadership par différents moyens :
      
      \begin{itemize}
       \item Organisation et coordination du travail
       \item Supervision et développement des individus
       \item Animation et conduite des équipes
      \end{itemize}
    
    \subsubsection{Gestion des ressources humaines}
      
      La gestion des ressources humaines est plus formalisée et a pour thèmes :
      
      \begin{itemize}
       \item Le recrutement et la sélection
       \item L'allocation et la planification des ressources humaines
       \item La classification des emplois et les rémunérations
       \item Le développement des employés et les formations
       \item L'évaluation des performances
       \item Le dialogue social et la communication
      \end{itemize}
      
  \subsection{Le modèle de Kolb}
  
  L'activité d'un manager dans les RH peut être représentée par le modèle de Kolb.
  
  % TODO : Insérer le modèle de Kolb
  
  \subsection{Structure générale du cours}
    Le cours se partage en deux axes.
    
    \paragraph{Axe historique}
      \begin{enumerate}
       \item L'administration du personnel
       \item L'école des relations humaines
       \item Le courant socio-technique
       \item La gestion stratégique des ressources humaines
      \end{enumerate}
    
    \paragraph{Axe thématique}
      \begin{enumerate}
       \item L'engagement et la motivation au travail
       \item Le pouvoir, l'autorité et le leadership
       \item La dynamique et les facteurs d'efficacité dans les groupes et équipes
       \item Les valeurs et la culture dans les organisations
      \end{enumerate}

\section{Panorama historique et rôles de la fonction RH}
	\subsection{Les étapes historiques}
	  \begin{itemize}
	   \item Contextes socio-économiques distincts
	   \item Modèles dominants d'organisation du travail et de pensée stratégique
	   \item Conceptions sous-jacentes de l'être humain
	   \item Rôles distincts pour la fonction RH
	   \item Champs de préoccupations et de pratiques RH
	  \end{itemize}
	  
	  \subsection{Les 4 rôles RH (D. Ulrich, 1993)}
	  
	  \begin{enumerate}
	   \item Expert administratif
	   \item Champion des employés
	   \item Agent de changement
	   \item Partenaire stratégique
	  \end{enumerate}

	  % TODO : Include la représentation des rôles RH Ulrich
	  
	  \begin{description}
	   \item[Expert administratif] : rôle centré sur la surveillance des contrats de travail et des règlements (processus et résultats)
	   \item[Champion des employés] : rôle centré sur la motivation des employés (la personne et son développement).
	   \item[Agent de changement] : rôle centré sur l'adaptation et l'évolution de l'organisation et des hommes (stratégies, approche socio-technique)
	   \item[Partenaire stratégique] : rôle centré sur la plus-value et les succès stratégiques de l'organisation (processus et résultats)
	  \end{description}
	  
	\subsection{L'administration du personnel}
	Les fondements de la fonction RH
	  \subsubsection{Le contexte socio-économique}
	  On est en pleine révolution industrielle (18-19ème siècle) dont les principales caractéristiques sont :
	  
	  \begin{itemize}
	   \item Progrès technologique : le machinisme, l'industrialisatio massive de la production ;
	   \item Disparition des corporations de métiers ;
	   \item Patron arbitraire, paternaliste ;
	   \item Avènement de l'entreprise capitaliste ;
	   \item Construction d'un marché de libre échange ;
	   \item Naissance des valeurs démocratiques républicaines ;
	   \item Dualisation de la société : bourgeoisie industrielle VS prolétariat
	   \item Perte d'autonomie et déqualification progressive des travailleurs
	   \item Conditions de travail misérables et précarisation économique de la classe ouvrière
	  \end{itemize}
	  
	  Cette période correspond également la naissance du mouvement ouvrier et du syndicalisme : nouvelle législation et apparition de la concertation sociale.

	  \subsubsection{Les modèles d'organisation du travail}
	  
	  \paragraph{L'organisation scientifique du travail (Taylorisme)}
	  \begin{description}
	   \item[Standardisation] : parcellisation et simplification des tâches (chronométrage des temps et mouvements).
	   \item[Distinction conception-exécution] : pour augmenter le rendement et la qualité de vie des ouvriers.
	  \end{description}
	  
	  \paragraph{Bureaucratie Wéberienne}
	  \begin{description}
	   \item[Formalisation et impersionnalité] : la règle remplace la tradition et l'arbitraire
	   \item[Centralisation et hiérarchisation] : structure hiérarchique conforme au principe d'unité de commandement (= un agent ne doit recevoir des ordres que d'un seul chef)
	  \end{description}
	  
	  \subsubsection{Les options stratégiques}
	  \begin{description}
	   \item[Fordisme] : gains en productivité et croissance assurés par une production et une consommation de masse.
	   \item[Planification rationnelle] : (one best way = la seule bonne façon de gérer) dans un environnement stable et peu compétitif. 
	  \end{description}

	  \subsubsection{La conception de l'homme}
	  \subsubsection{Conception de la fonction RH}
	  \subsubsection{Pratiques héritées de la RH}
	\subsection{L'école des relations humaines}
	\subsection{L'approche socio-technique}
	\subsection{La gestion stratégique des RH}
	
		\subsubsection{L'approche socio-technique}
		\subsubsection{La gestion des ressources humaines}

\section{Culture organisationnelle}
	\subsection{Historique}
	\subsection{Le modèle de l'oignon d'Edgar Schein}
	
	Le modèle d'Edgar Schein est une conceptualisation de la notion de culture d'entreprise via une approche anthropologique.
	
	Pour Edgar Schein, la culture organisationnelle se définit comme \og{} un ensemble de prémisses et de croyances partagées que le groupe a appris au fur et à mesure qu'il a résolu ses problèmes d'adaptation externe et d'intégration interne, qui a fonctionné suffisamment bien pour qu'il soit considéré valide, et par conséquent est enseigné aux nouveaux membres comme la manière appropriée de percevoir, de penser et de ressentir par rapport à ces problèmes \fg{} \cite{schein2010}
	
	  \subsubsection{Un phénomène multi-niveaux}
	  
	  Edgar Schein divise la culture d'une entreprise en plusieurs niveaux :
	  
	  \begin{multicols}{2}
	  
	  Les \textbf{artéfacts} sont les aspects visibles de la culture : la manière de s'habiller, les logos, le jargon, l'humour,... Ils sont faciles à identifier par une démarche ethnographique mais il est difficile d'en tirer une signification sans analyser les autres niveaux de l'oignon.
	  
	  Les \textbf{valeurs} sont les stratégies, les objectifs et philosophies choisies de manière consciente et qui sont diffusées par la direction et le management de l'entreprise : compétitivité, solidarité, adaptation au changement, stabilité, etc.
	  
	  Les \textbf{normes de pensée et d'action} correspondent aux routines comportementales, habitudes, modèles d'action, rituels et schémas cognitifs d'inteprétation des évènements. Elles sont directement déterminées par les valeurs sous-jacentes.
	  	  
	  \begin{center}
	    \includegraphics[width=\linewidth]{culture_orga_niveaux.png}
	  \end{center}
	  Les \textbf{postulats fondamentaux} ou prémisses sont les croyances qui sont l'essence de la culture. Ces prémisses sont difficiles à discerner car elles opèrent au niveau de l'inconscient. Elles portent sur des questions telles que la nature de l'homme, le rapport au temps, la notion de vérité, etc. Elles ne sont quasiment jamais remises en cause.
	  
	  Les artéfacts découlent des valeurs et les valeurs découlent des postulats fondamentaux.
	  
	  \end{multicols}
	  
      \subsection{Le modèle des valeurs concurrentes de Quinn}
	Ce modèle a été développé à l'origine pour décrire les valeurs sous-jacentes aux critères d'efficacité organisationnelle.
	
	Il caractérise les cultures organisationnelles selon 2 dimensions : le contrôle VS la flexibilité et les relations (interne) VS les résultats (externe).
	
	\subsubsection{Un phénomène multi-dimensionnel}
	
	  Quinn définit 4 grands types de cultures :
	  
	  La \textbf{culture de soutien} qui privilégie la coopération, la participation, l'attachement à l'entreprise, la communication interpersonnelle. C'est une culture de \og{} collaborateurs \fg{} qui peut se retrouver notamment dans les PME familiales.
	  
	  La \textbf{culture des règles} qui est très respectueuse des procédures, tout est écrit, tracé, standardisé. La communication est essentiellement descendante dans l'organigramme, c'est une culture \og{} d'organisateurs \fg{}. Cette culture peut se retrouver dans les administrations publiques.
	  
	  La \textbf{culture des buts} privilégie les objectifs de performance et la rationalisation des processus en vue des objectifs. C'est une culture de \og{} compétiteurs \fg{}. 
	  
	  La \textbf{culture de l'innovation} qui privilégie la créativité, l'ouverture au changement, l'expérimentation et l'adaptation permanente. La communication est aussi peu formalisée que possible pour casser la hiérarchie stricte, c'est une culture \og{} d'innovateurs \fg{}.
	  
	  Ce modèle peut se superposer à celui d'Ulrich pour tracer un historique des pratiques de ressources humaines.
	
	\subsection{La théorie des dimensions culturelles (1991)}
	
	Une organisation (de taille conséquente) n'est pas liée à une seule culture, il existe des sous-cultures. Il y a les sous-cultures par département, par métiers, par âge, par affinités,...
	
	  \subsubsection{Un phénomène hétérogène}
	  
	  Le modèle de Hofstede a pour but d'étudier les interactions entre les cultures en fonction de certaines dimensions en leur attribuant des scores de 1 à 120.
	  
	  Il met en avant 4 dimensions :
	  
	  \begin{enumerate}
	   \item Distance au pouvoir
	   \item Evitement de l'incertitude
	   \item Masculinité contre féminité
	   \item Individualisme contre collectivisme
	  \end{enumerate}
	  
	  La \textbf{distance au pouvoir} est un indice d'acceptation de pas avoir de pouvoir par les membres les plus éloignés de la direction dans l'organigramme.
	  Un indice faible signifie que les membres souhaitent une gestion démocratique et se considèrent à égalité avec les autres, un indice élevé indique que ceux qui ont le moins de pouvoir acceptent leur condition et sont forts soumis au pouvoir.
	  
	  L'\textbf{évitement de l'incertitude} correspond au degré de tolérance d'une société pour l'incertitude/l'ambiguité. Les sociétés avec un indice faible sont ouvertes aux changements, disposent de moins de règles et de lois et les directives y sont plus souples.
	  
	  La \textbf{masculinité contre la féminité} correspond au niveau d'importance accordée aux valeurs masculines (assurance, ambition, pouvoir, matérialisme) et aux valeurs féminines (égalité, relations humaines).
	  
	  L'\textbf{individualisme contre le collectivisme} correspond au degré auquel les individus sont intégrés aux groupes. Une culture individualiste donne de l'importance à l'initiative privée et la réussite d'objectif personnels, une culture collectiviste met en avant le bien-être du groupe, la loyauté, l'intérêt collectif avant l'intérêt personnel.
	  
	  Il est possible de croiser les dimensions entre elles.
	  
	  L'analyse pour la culture belge pourrait donner ceci : une distance au pouvoir plutôt forte, un évitement de l'incertitude fort, une légère prédominance masculine et une culture plutôt individualiste.
	  
	  \begin{figure}[!ht]
	   \begin{center}
	    \includegraphics[width=0.65\linewidth]{hofstede_abbr.png}
	    \caption{Abréviations pour les pays et régions étudiés}
	   \end{center}
	  \end{figure}
	  
	  \begin{figure}[!ht]
	   \begin{center}
	    \includegraphics[width=0.65\linewidth]{hofstede_indivi_collect.png}
	    \caption{Position des 50 pays et 3 régions sur la distance au pouvoir et l'individualisme-collectivisme}
	   \end{center}
	  \end{figure}
	  
	  \begin{figure}[!ht]
	   \begin{center}
	     \includegraphics[width=0.65\linewidth]{hofstede_power_masc.png}
	     \caption{Distance au pouvoir contre masculinité pour 50 pays et 3 régions}
	   \end{center}
	  \end{figure}

	  \begin{figure}[!ht]
	   \begin{center}
	     \includegraphics[width=0.65\linewidth]{hofstede_masc_indivi.png}
	     \caption{Position de 50 pays et 3 régions sur les dimensions masculinité-féminité et individualisme-collectivisme}
	   \end{center}
	  \end{figure}
	  
	  \begin{figure}[!ht]
	   \begin{center}
	    \includegraphics[width=0.65\linewidth]{hofstede_masc_incertitudes.png}
	    \caption{Position de 50 pays et 3 régions sur les dimensions masculinité/féminité et évitement de l'incertitude}
	   \end{center}
	  \end{figure}
	  
	  \begin{figure}[!ht]
	   \begin{center}
	    \includegraphics[width=0.65\linewidth]{hofstede_incertitudes_indivi.png}
	    \caption{Position de 50 pays et 3 régions sur les dimensions d'évitement d'incertitude et individualisme-collectivisme}
	   \end{center}
	  \end{figure}

	  \FloatBarrier % Force les flottants à se placer ici


	\subsection{Un phénomène dynamique}
	\subsection{Méthodologies d'étude}
	\subsection{Les répercussions}

\section{Leadership}

\section{Dynamique de groupe}

\section{Motivation au travail}

\section{Conclusion}

\bibliographystyle{plain}
\bibliography{biblio}


\end{document}