\documentclass[a4paper, 12pt]{article}
\usepackage[utf8]{inputenc}
\usepackage[T1]{fontenc}
\usepackage[french]{babel}
\usepackage{geometry}
\usepackage{times}
\usepackage{multicol}
\usepackage{titling}
\usepackage{graphicx}

\newcommand{\HRule}{\rule{\linewidth}{0.5mm}}

\geometry{hmargin=2.5cm,vmargin=2.5cm}

\begin{document}
\begin{titlepage}
\begin{center}

% Upper part of the page. The '~' is needed because \\
% only works if a paragraph has started.
\includegraphics[width=0.15\textwidth]{./sedes_ucl.png}~\\[1cm]

\textsc{\LARGE Université Catholique de Louvain}\\[1.5cm]

\textsc{\Large Travail pratique de Management humain}\\[0.5cm]

% Title
\HRule \\[0.4cm]
{ \huge \bfseries Les rôles de la fonction Ressources Humaines\\[0.4cm] }

\HRule \\[1.5cm]

% Author and supervisor
\noindent
\begin{minipage}[t]{0.4\textwidth}
\begin{flushleft} \large
\emph{Auteurs:}\\
Jérôme \textsc{Claeys} \\
Johannes \textsc{Croe} \\
Georgi \textsc{Frolov} \\
Olivier \textsc{Kerremans} \\
John-John \textsc{Ketelbuters} \\
Tanguy \textsc{Meert}

\end{flushleft}
\end{minipage}%
\begin{minipage}[t]{0.4\textwidth}
\begin{flushright} \large
\emph{Professeurs :} \\
Nathalie \textsc{Delobbe} \\
Thi Thu Thao \textsc{Lê}
\end{flushright}
\end{minipage}

\vfill

% Bottom of the page
{\large \today}

\end{center}
\end{titlepage}

\newpage
\tableofcontents
\linespread{1}
\newpage

\section{Introduction}

Dans le cadre du cours de Management Humain, il nous a été demandé de réaliser un travail illustrant la fonction RH dans son ensemble. Par groupe d'étudiants, nous avons eu l'occasion d'assister à une table ronde organisée avec des DRH de différentes entreprises. L'objectif de cette table ronde était bien entendu de comprendre les tensions qui peuvent exister entre les différents rôles de la fonction RH ainsi que de confronter l'orientation et le point de vue de plusieurs experts du domaine. Nous avons essayé de retirer un maximum d'informations de leurs expériences, parcours professionnels au sein des ressources humaines . Nous nous sommes efforcés de garder un comportement critique et une démarche objective lors de la réalisation du travail. On s'est employé à illustrer nos propos par des exemples le plus souvent tirés de la table ronde. Réaliser ce travail nous a mis à l'épreuve : nous avons expérimenté le phénomène de dynamique de groupe. Nous avons procédé à une série d'entretien, réunion réguliers et avons répartis les sections du travail par sous-groupe de deux personnes. Nous étions six, nous nous sommes concentrés d'abord séparément par groupe de deux sur les trois premières questions et avons par la suite procédé à une mise en commun. Enfin, après avoir exploré le modèle théorique en détail, nous nous sommes penchés sur ses limites et nous sommes interrogés sur l'avenir de la fonction des ressources humaines.

\section{Identification des rôles, missions, défis et pratiques des DRH}

\subsection{Stephan Van Lierde}

Stephan Van Lierde est depuis 5 ans directeur des ressources humaines à la RTBF. Juriste de formation, puis MBA, il travaille dans les ressources humaines depuis le début de sa carrière. Avant d'entrer à la RTBF, il a travaillé dans divers secteurs ; les assurances, les télécommunications, l'audit informatique et le secteur pharmaceutique.

La RTBF (Radio et Télévision Belge Francophone) compte environs 2000 travailleurs. Hormis 3 chaînes de télévision et 5 de radio, il y a aussi toute une division web développée récemment. Environ 2000 personnes y travaillent, 350 journalistes, 500 personnes pour la publicité et l'archivage, 500 " intermittents du spectacle " qui travaillent quand l'actualité le nécessite, 750 ingénieurs et une quarantaine de personnes actives sur les réseaux sociaux.

La RTBF est fortement marquée par l'évolution de la technologie qui représente à elle seule un triple défi. Premièrement, la technologie offre des possibilités nouvelles quant à l'amélioration du service offert aux personnes et à la diminution des besoins de l'entreprise en personnel. À titre d'exemple, depuis 10 ans, le nombre de travailleurs est passé de 3000 à environ 2000 et la production a augmenté de 25 \%. Un défi majeur de la RTBF est donc de saisir les opportunités offertes par la technologie. Deuxièmement, le développement technologique bouleverse également les habitudes des consommateurs, auxquelles la RTBF se doit de répondre notamment par une grande transversalité dans l'offre, c'est à dire proposer chaque information à la fois sur internet, à la télévision et à la radio. Troisièmement, la technologie a pour effet de rendre certaines fonctions obsolètes. Il faut donc remanier la classification des fonctions avec pour objectifs d'une part d'arriver à une description suffisamment précise de la fonction et d'autre part de faire en sorte que cette classification reste valable le plus longtemps possible. Le rôle de M\up{r} Van Lierde est ici de convaincre les corporations réticentes à s'adapter au développement technologique que cette évolution est nécessaire. Les syndicats refusent par exemple qu'un journaliste puisse filmer ou faire un montage vidéo. Cela passe par un changement de la culture des gens. Il faut aussi exploiter la technologie au niveau des ressources humaines pour améliorer la communication au sein de l'organisation. Le but est de fournir de l'information très bas hiérarchiquement sans passer par des échelons intermédiaires. Quand l'information doit être rapidement relayée au grand public, le chef doit rapidement avoir accès à tous ses collaborateurs. Il s'agit pour les ressources humaines de fournir des outils et des méthodes permettant d'anticiper l'urgence. Notons qu'accompagner ce changement est aux yeux de M\up{r} Van Lierde sa plus value la plus importante au sein de l'organisation. Selon lui, pour être crédible dans cette démarche, il faut bien comprendre le business.

En dehors de la technologie, nous avons repéré plusieurs autres défis. D'abord, le business plan est particulier. En partant chaque matin d'une feuille blanche, il faut présenter un JT quelques heures plus tard. De ce fait, il est important de travailler avec des personnes créatives, qu'il faut d'abord recruter et former. La mission est d'abord de chercher parmi une grande variété de profils des personnes, qui moyennant formation, pourraient convenir pour un poste à pourvoir. Il est ici important d'avoir une bonne image en tant qu'employeur pour attirer les talents dont on a besoin. Pour la formation, la RTBF a créé une académie mais celle-ci ayant beaucoup de succès, il n'y a pas de place pour tout le monde. Il y a donc une sélection, opérée par M\up{r} Van Lierde. Le rôle important de la technologie s'illustre encore sur ce point puisque cette académie recourt à l'e-learning depuis maintenant un an.

On trouve un autre défi dans la mise au point d'une douzaine de systèmes de fonctionnement. Ceux-ci ont pour but d'objectiver la performance des différents métiers de l'organisation. Le problème visé ici est que certaines personnes, ayant des contributions parfois très inégales sont payées de la même manière. Les travailleurs acceptent mal d'être traités inéquitablement et se chargent parfois d'exclure d'une manière ou d'une autre ceux qui contribuent moins, ce qui peut très mal se passer. Objectiver ce qu'on attend des collaborateurs permet d'instaurer davantage d'équité et de limiter le risque d'exclusion basée sur des critères subjectifs. On peut aussi situer la contribution d'une personne par rapport à ce qu'on attend d'elle, ce qui aide quand des décisions de licenciement doivent être prises. M\up{r} Van Lierde donne généralement une chance de s'améliorer aux personnes les moins efficaces avant de les licencier si aucun effort n'est fourni. Ce système est particulièrement utile car les managers ne s'occupent pas de ce genre de décisions difficiles, et les confient aux ressources humaines. Mr Van Lierde veille à la mise au point de ces systèmes et s'engage à former les managers à prendre ces décisions.

Au sein de l'équipe de direction, le DRH est celui qui est le plus tourné vers les travailleurs. Son rôle est de participer à la décision en étant réaliste à l'égard des facteurs liés aux ressources humaines. Il devra faire attention aux indicateurs clés de performance (KPI) comme l'audience et plus particulièrement en tant que DRH, aux chiffres d'absentéisme. Mr Van Lierde est tourné vers l'humain mais pas au sens social, il ne prétend pas faire le bonheur des travailleurs et n'est pas assistant social.

\subsection{Hugues Botman}

Hugues Botman est responsable des Ressources Humaines chez Pfizer pour le Benelux et l'Irlande, ce qui représente environ 5200 personnes. Il y travaille depuis maintenant 16 ans. Celui-ci a des responsabilités belges ainsi que dans l'ensemble de l'Europe. 

Pour Hugues Botman, le rôle des Ressources Humaines est de faire le lien entre l'économique et l'humain. Par exemple, du point de vue humain, il faut augmenter le bien-être des consommateurs en leur fournissant un médicament efficace et peu cher et de l'autre coté, du point de vue économique, il y a le fait que l'entreprise cotée en bourse doit atteindre certains objectifs.

L'organisation RH chez Pfizer compte un peu moins de 1000 employés. Il existe 3 grands pôles dans cette organisation suivant le modèle Ulrich:
En premier lieu, viennent les centres d'expertise/excellence. Ce pôle consiste en de petits groupes qui définissent les politiques qui vont s'appliquer à l'ensemble de l'entreprise. On y  trouve deux domaines. On y trouve  le domaine du \textit{total rewatching} dans lequel ils gèrent la politique et stratégie en matière de rémunération sur le plan mondial. Le 2\up{ème} domaine est tout ce qui touche la gestion des talents (Learning development, searching development,etc). Tous ces processus sont définis de manière centrale et s'appliquent à l'ensemble de l'entreprise.

Ensuite, il y a les \textit{global business Partners}. Ce sont des groupes de personnes qui travaillent avec les patrons des différentes divisions, départements de l'entreprises. Ils sont chargés de définir en collaboration étroite avec la business unit leurs besoins stratégiques. Ils vont également élaborer avec eux leur business plan, en particulier la partie humaine et organisationnelle de celui-ci (1-2 pers par business unit).

Enfin vient l'organisation des ressources humaines locales reprenant 85-90\% des effectifs des RH. C'est dans ce pôle là que M\up{r} Botman travaille. Ce sont les DRH issus des différentes divisions qui sont directement sur le terrain. Ils sont aussi responsables du \textit{HR service delivery model} et disposent d'une plate-forme informatique leur permettant d'effectuer leur tâches. Cette plate-forme contient énormément d'applications de type services à la fois pour les collègues et les managers. Elle permet une série de transactions ou informations dont ils ont besoin pour gérer leur propre carrière et développement.  Le rôle de M\up{r} Botman (Irlande et Benelux) est de s'appuyer sur les outils développés par les centres d'excellence et de bien comprendre les stratégies des différents business pour les mettre correctement en œuvre. Il a donc un rôle de people manager (40 personnes) sur les différents sites. Celui-ci doit bien comprendre les besoins du business et allouer les ressources là ou elles sont nécessaires.

Une des missions générales des RH est de contribuer au maximum à la réduction des coûts. Il y a eu récemment une réduction des coûts et des effectifs des Ressources Humaines de près de 60\%. Cette diminution des effectifs est due à une diminution du chiffre d'affaire de 65 à 50 milliards de dollars en 5 ans, elle-même causée par le fait que certains médicaments arrivent au terme de leurs brevets.

 Une autre mission se situe au niveau de l'adaptation du système d'évaluation. L'année dernière, Pfizer est passé, au niveau du système d'évaluation des performances, vers un système standardisé et plus qualitatif. Il s'agit d'un système sans étiquette avec la volonté de privilégier l'entretient et le feedback . Chez Pfizer, un accompagnement du personnel est mis en place grâce à des formations, il y a une volonté de maintenir l'employabilité tout au long de leur carrière. M\up{r} Botman a donc également un rôle de supervision concernant ce sujet. Cela permet un taux de reclassement de quasi 100\% lorsqu'ils doivent se séparer d'un employé.

Récemment, Pfizer a dû faire face à un défi qui consiste en un transfert de certains rôles des RH vers le Management, ce qui a permis la réduction des coûts et des effectifs mentionnée plus haut. Le management est formé et encadré afin de pouvoir faire face à ces nouvelles responsabilités. Les RH doivent donner des outils au Management afin d'effectuer ces nouvelles tâches.
Un autre défi que doit relever Pfizer, est de savoir concilier la technologie avec la fonction des ressources humaines. En effet il faut tenter de garder un maximum de contacts réels avec les employés étant donné que ceux-ci sont souvent remplacés par la technologie (on pense notamment à la vidéoconférence).

Finalement, une pratique chez Pfizer à été mise en place, consistant en un système d'échange entre un ou deux membres de la direction ainsi que quelques travailleurs choisis totalement au hasard, indépendemment de leur rôle, fonction, ancienneté dans l'entreprise. Cette petite discussion permet de faire ressortir des problèmes présents au sein de l'entreprise, et tout cela dans une ambiance décontractée.


\subsection{Marie-Pierre Saint-Viteux}

Par son expérience, Marie-Pierre Saint-Viteux affirme avoir exercé beaucoup de fonctions avant d'arriver dans les ressources humaines. En effet, selon elle il est indispensable d'avoir approché un large éventail d'activités pour s'adapter au mieux à cette fonction au sein de son entreprise.

Marie-Pierre Saint-Viteux est directrice des ressources humaines pour l'activité « Constructs Equipments » (excepté le secteur du Développement), qui représente environ 5000 personnes à travers le monde, mais aussi du quartier général de Volvo se situant à Bruxelles. Cette fonction nécessite de travailler avec le monde entier, et ceci 24h/24, impliquant donc flexibilité et omniprésence. De manière plus indirecte, elle sous-entend qu'il est impératif d'être ouvert sur les différentes cultures, visions et points de vue. Elle affirme aussi la nécessité de rendre le cadre de travail le plus agréable possible étant donné la majeure partie du temps qu'y passe le travailleur dans le courant de sa vie.

Un de ses principaux défis est d'être confrontée à l'environnement instable et au changement perpétuel de l'ensemble du monde (social, politique, économique, technologique'). Ces différents facteurs impliquent des cycles d'activités très courts et une prévisibilité quasi nulle de demain,  des scénarios "ups and downs"  tel que les qualifie M\up{me} Saint-Viteux. Le défi majeur est donc d'être le plus réactif possible face aux besoins du marché, posséder dans ses réserves les personnes avec les compétences requises et disponibles au bon endroit et au bon moment.    
						
	Cette approche est la conséquence de ces cycles instables et imprévisibles. On s'y adapte donc au mieux avec les ressources humaines constamment prêtes à intervenir et en attente de leur déploiement, \textit{Flexible Workforce} comme dit par M\up{me} Saint-Viteux. Effectivement, Volvo travaille avec 30\% d'effectif flexible, selon la richesse ou le manque d'activité. Cette rapidité du changement implique aussi un autre défi, celui d'accompagner ses employés vers le futur, vers un nouveau cycle d'activités. Dans cette transition, il est nécessaire de pouvoir former rapidement l'effectif aux nouvelles compétences, le tout devant se faire dans un esprit et une mentalité positive d'acceptation du changement, chose qui peut parfois être difficile et mal vue.
 
Elle souligne aussi la disparition des postes de long terme et des possibilités de carrière allant du bas de l'échelle pour atteindre la place de CEO. Aujourd'hui, selon elle, et pour cause l'avenir incertain, les entreprises requièrent premièrement de la flexibilité chez leurs employés. Pour les entreprises elles-mêmes et leurs décisions stratégiques, il apparaît une pression poussant à la diversification afin de répondre au mieux, et de manière la plus personnalisée qui soit au besoin du client. Ceci demande d'élargir son éventail d'activités  et donc de trouver les personnes compétentes avec les compétences nécessaires, voire même les talents, afin d'occuper ces nouveaux postes. C'est une des tâches difficiles réservées aux ressources humaines. « Le niveau humain est à la base de toute réalisation », tel que dit par M\up{me} Saint-Viteux.

Un des défis de 2014 fut l'acquisition d'une entreprise de 500 personnes. Cela commence par une \textit{due diligence}, c'est à dire une analyse préalable de l'entreprise par son acquéreur potentiel. Par après, on retrouve un rôle de tutelle auprès des ressources humaines de l'entreprise en question ce qui demande une certaine formation, un apprentissage avec un processus d'accompagnement interne afin de permettre une bonne coordination et un bon fonctionnement par la suite.
M\up{me} Saint-Viteux consacre 30\% de son temps à la tâche de \textit{Compensations and Benefits}, poste qui se charge de l'admission et du calcul des salaires. Une difficulté majeure est leur harmonisation. En effet, par sa présence dans le monde entier et ses quelques 110 000 travailleurs, Volvo se doit de trouver la bonne parité et par ce fait, de savoir distribuer les salaires de façon équitables selon les fonctions occupées et les différents pays.                   	            	     
Une activité parallèle y est reliée : le processus motivationnel dans lequel Volvo doit  trouver un équilibre afin de garantir sa propre stabilité/croissance financière d'une part, et d'autre part, de promouvoir le travail auprès de ses employés dans la réalisation de ses objectifs et ce, par exemple, par l'octroi de bonus.

Le modèle de service utilisé par Volvo est le \textit{Service Delivery Model}, celui-ci regroupe 3 grands pôles avec au sommet le \textit{Center of Expertise}, le centre des experts en la matière. Deuxièmement, on trouve le \textit{Service Center} qui s'occupe des tâches plus routinières. Troisièmement, il y a le \textit{Head Chart business Partner}, pôle constitué des responsables en ressources humaines qui secondent les managers afin de réaliser les objectifs de Volvo.  

Par ailleurs, Volvo possède en ses ressources un système de base de données d'organisation et de gestion des travailleurs les répertoriant par leurs compétences. Sont aussi répertoriées les compétences requises pour chaque poste en question. Ceux-ci se veulent actuellement de plus en plus polyvalents. Cela permet de constater les écarts entre les compétences effectives et le niveau de compétences requis.  Avec ce système, il est donc possible de former au mieux les travailleurs pour combler leurs lacunes. Ce processus permet aussi d'effectuer des analyses et des statistiques sur les ressources humaines de l'entreprise telles que les compétences et capacités de l'effectif. Ces analyses peuvent par ailleurs offrir une aide aux décisions stratégiques à entreprendre. Une autre de ses ressources est la \textit{Volvo University} qui se charge de la formation des employés et éventuellement du comblement des lacunes. Cela ce fait grâce à un programme de qualité disponible en e-learning. À noter que pour certains postes, l'université a mis en place un programme de \textit{Succession Planning} pouvant aller jusqu'à former trois successeurs prétendants au poste en question.

\section{Comparaison des rôles RH}

\textit{Lors des interventions de nos invités, nous avons pu remarquer des critères  partagés au niveau des rôles et  fonctions.}

D'abord, on constate que les trois entreprises doivent faire face à un environnement marqué par le changement. Ce changement, qu'il soit d'ordre politique, social, économique ou plus particulièrement d'ordre technologique, a pour conséquence une instabilité au niveau de l'activité des entreprises, ainsi qu'une incertitude vis-à-vis du futur. Cette mutation  impose aux 3 entreprises d'être réactives afin de s'adapter au mieux aux nouvelles tendances, en particulier aux besoins des clients. La RTBF et Pfizer semblent davantage marquées par l'avancée technologique. En effet, M\up{r} Van Lierde considère l'accompagnement du changement technologique comme son rôle principal. M\up{r} Botman parle également de l'accompagnement des changements organisationnels tels que les restructurations comme un rôle important. Chez Volvo, M\up{me} Saint-Viteux insistera davantage sur la flexibilité de la main d'œuvre nécessaire pour réagir rapidement à ces nouveautés. M\up{me} Saint-Viteux doit assurer en permanence la disponibilité de personnes compétentes pour faire face aux imprévus. Une flexibilité d'au moins 30\% de la main d'œuvre à la hausse comme à la baisse doit être prévue pour cela. Ce besoin de flexibilité se retrouve aussi à la RTBF où M\up{r} Van Lierde fait appel à certains travailleurs selon le besoin dicté par l'actualité. 

On peut aussi remarquer que tous les trois insistent sur l'importance de bien comprendre le business et la stratégie afin de la mettre en œuvre au mieux. C'est pourquoi il est important de bien connaître les fonctions exercées au sein de l'entreprise. Pour M\up{me} Saint-Viteux, la meilleure façon de développer cette compréhension est d'avoir soi-même occupé une grande variété de fonctions auparavant. M\up{r} Van Lierde, ayant travaillé dans beaucoup d'autres secteurs, n'a pas exercé lui-même les fonctions présentes à la RTBF. Il va donc améliorer sa connaissance du fonctionnement de l'entreprise en se rendant sur le terrain au moins deux fois par mois.

	Un autre rôle assuré par les 3 intervenants consiste à rechercher et si nécessaire à former des talents et des esprits créatifs. On notera que les 3 entreprises bénéficient toutes d'une bonne notoriété, ce qui facilite ce rôle de recrutement des DRH. Concernant la formation à la RTBF comme chez Volvo, on observe un recours important à l'e-learning mais aussi en moindre mesure utilisé par Pfizer. Il apparaît ici un double effet que doivent pouvoir mettre en relation les DRH. Le premier, d'ordre budgétaire, est d'avoir l'effectif le plus réduit mais le plus performant et polyvalent possible. On retrouve cette démarche chez Pfizer. Après une diminution du chiffre d'affaire, une diminution de près de 60\% du personnel a été opérée dans le département des ressources humaines, en partie grâce à un transfert de rôle du département RH vers le management. À la RTBF, la technologie a permis de produire 25\% en plus avec 35\% de personnel en moins. Deuxièmement, face aux besoins toujours nouveaux et de plus en plus diversifiés des clients, il est nécessaire de trouver les bonnes personnes, rôle du DRH et de son département. Ainsi, afin de refléter la diversité des clients, on recherche la diversité dans la main d'œuvre chez Volvo, ce qui permet d'être plus proche des besoins des clients. Chez Pfizer, les traitements médicaux produits sont de plus en plus individualisés, c'est à dire qu'ils ciblent un plus petit nombre de patients qu'auparavant. Enfin, à la RTBF, l'information est fournie sur tous les supports pour informer le plus grand nombre.	

Il faut aussi souligner le défi qu'ont les DRH en ce qui concerne la gestion des salaires. Face à la diversité des cultures et des corps de métiers présents dans ces entreprises, la question des rémunérations se pose. En effet, cette difficulté à savoir quantifier de façon monétaire les différentes tâches représente un réel défi pour les DRH. Cette question se pose d'autant plus lorsque l'entreprise est présente dans plusieurs pays, comme Volvo et Pfizer. Chez Pfizer, M\up{r} Botman a accompagné la récente révision de la politique de rémunération. Chez Volvo, Marie-Pierre Saint-Viteux œuvre avec son équipe à une standardisation des rémunérations d'une part et d'autre part à l'usage de la rémunération comme une motivation complémentaire. Cela passe par la recherche d'un équilibre entre rémunération fixe et variable. 

Cette démarche n'est complète que si l'on peut définir clairement la fonction des différents collaborateurs. Donc, dans un souci d'équité, les entreprises développent des moyens d'objectivation de ce qu'elles attendent des employés. En ce qui concerne la RTBF, cela passe par la mise au point des systèmes de fonctionnement dont nous avons parlé plus haut. Chez Volvo, tous les travailleurs dans le monde sont évalués selon un processus similaire et soumis aux mêmes types de questions. Chez Pfizer, l'évaluation est basée sur un dialogue.

Notons également chez Pfizer et à la RTBF un transfert des compétences des ressources humaines au management. Les deux DRH ont alors un rôle de formation auprès du management.	
Un élément qui différencie le rôle de DRH à la RTBF, probablement lié à sa plus petite taille comparé aux deux autres entreprises, est que Stephan Van Lierde est un membre du comité de direction.

Enfin, les DRH ont pour mission primaire de rendre le cadre de travail agréable et propice à la performance de leurs employés. Par ailleurs il n'est pas de leur but de rendre les travailleurs heureux car cela ne dépend pas uniquement du travail mais aussi de la situation personnelle de chaque travailleur. C'est donc bien seulement dans le contexte travail que les ressources humaines agissent afin d'assurer leur bien-être au travail.


\section{Parallèle avec le modèle théorique}
 
 \subsection{Description du modèle} \footnote{Cette section est inspirée de CONNER et ULRICH, 1996}
Avant de commencer à faire des parallèles avec un modèle théorique, il convient d'abord de le définir et de l'expliquer. 
Le modèle théorique des rôles de la fonction RH est un cadre théorique développé par Dave Ulrich en 1993. Il découpe la fonction en 4 rôles distincts, répartis sur deux axes. L'axe horizontal oppose les perspectives opérationnelle et stratégique. Opérationnel signifie être centré sur le présent. Stratégique signifie être davantage attentif au futur, à l'adaptation à l'environnement par exemple. L'axe vertical reflète l'arbitrage entre une attention aux personnes et une attention au processus. La juxtaposition de ces deux axes définit 4 rôles différents.

Le premier rôle de \textit{champion des employés} consiste à faire en sorte que les employés fassent leur devoir au sein de l'entreprise. Pour cela, le DRH doit jouer un rôle de psychologue et s'assurer de la motivation ainsi que la bonne contribution des travailleurs. Il devra donc réussir à concilier les intérêts économiques avec les intérêts personnels des employés.
Le deuxième rôle d'\textit{expert administratif}  illustre le rôle traditionnel des ressources Humaines. Il s'agit entre autres de l'administration des contrats de travail ou encore de la gestion des systèmes de paie et d'incitation au travail. Le DRH s'occupe également des conflits qui pourraient apparaître au sein du personnel.
Le troisième rôle s'intitule \textit{partenaire stratégique}. Il veille l'alignement entre la stratégie des ressources humaines et la politique de l'entreprise. Le DRH joue désormais un rôle de gestionnaire et de membre de la direction. En déployant les ressources humaines, il s'emploie à créer de la valeur ajoutée pour l'entreprise et l'aide par la même occasion à atteindre ses objectifs économiques et stratégiques.
Enfin on peut associer le rôle \textit{Agent de changement} à un rôle d'accompagnement. Dans ce cas ci, le DRH va chercher à être un bon gestionnaire de projet. Son but sera de faciliter les transitions internes à l'entreprise. A titre d'exemple, on pense à des réorganisations ou des introductions de nouvelles manières de travailler ou encore aider les membres à s'adapter à un nouvel environnement compétitif.

\subsection{Application du modèle}
En ce qui concerne la RTBF, nous remarquons qu'un des rôles principaux que Stephan Van Lierde doit remplir est celui d'agent de changement. En effet, avec l'avènement de nouvelles technologies, il doit accompagner son entreprise et ses membres dans la transition qu'amène la technologie. Il y a une toute nouvelle manière de travailler, de nouveaux moyens de communication mais surtout une offre qui devient transversale et qui change complètement. C'est tout un projet de passer d'une offre standard combinant télévision et radio à une offre transversale également disponible en intégralité sur internet. Le rôle de M\up{r} Van Lierde est d'accompagner et de mener à bien ce projet en opérant sur les ressources humaines. Tout ceci passe par un changement de la culture des travailleurs, ils doivent accepter qu'un employé ne soit plus forcément cantonné à une seule fonction. 

On peut également relever que M\up{r} Van Lierde occupe le rôle de partenaire stratégique. Au sein de l'équipe de direction, c'est lui qui est le plus axé vers l'humain mais pas dans le sens du social. Il ne travaille pas forcément au bonheur des travailleurs mais s'assure simplement que les ressources humaines permettent d'atteindre les objectifs clés de performance de la RTBF.  De plus, au niveau formation, la RTBF a créé une académie connaissant beaucoup de succès qui permet une spécialisation des employés permettant à l'entreprise d'atteindre ses objectifs économiques. 
On peut tout de même remarquer qu'en tant que bon DRH, M\up{r} van Lierde a également quelques traits que l'on peut assimiler au rôle de champion des employés. Il se soucie de ses employés et cherche  à assurer la motivation et la bonne contribution en  donnant généralement une chance de s'améliorer aux personnes les moins efficaces avant de les licencier si aucun effort n'est fourni. Finalement, il se rend sur le terrain au moins deux fois par mois et est à la charge de la gestion des carrières.

Au niveau de l'entreprise Pfizer, on peut clairement identifier les rôles de champion des employés et de partenaire stratégique. En effet, Pfizer est récemment passé à un système d'évaluation sans étiquette, un système plus qualitatif où l'accent est mis sur le feedback. Il y a une volonté de suivre le personnel, de s'assurer que celui-ci se sente bien afin qu'il donne le meilleur de lui-même. Cela passe également par les petites réunions aléatoires que nous avons mentionnées plus haut . On distingue très clairement cette envie de jouer les psychologues afin de créer une ambiance de travail conviviale.
Du point de vue partenaire stratégique,  M\up{r} Botman s'appuie sur les outils développés par les centres d'excellence et tente de bien comprendre les stratégies des différents business unit pour bien les mettre en œuvre.

Marie Pierre Saint Viteux sera elle plutôt qualifiée d'agent de changement, de partenaire stratégique et de championne des employés. D'un coté, celle-ci se voit confrontée au challenge d'un environnement instable et au changement perpétuel de l'ensemble du monde (social, politique, économique, technologique). Ces différents facteurs impliquent des cycles très, voire trop courts d'activités et une prévisibilité quasi nulle de demain. Cette fonction implique qu'il est nécessaire de travailler avec le monde entier, et ceci 24h/24, impliquant donc flexibilité et omniprésence. De manière plus indirecte, elle sous-entend qu'il est impératif d'être ouvert sur les différentes cultures, visions et points de vue.  De plus, face a cet environnement changeant, on peut la qualifier de partenaire stratégique car, selon elle, le défi majeur est  d'être le plus réactif possible face aux besoins du marché. Il faut avoir à disposition les personnes avec les compétences requises disponibles au bon endroit et au bon moment.
Finalement, on pourrait la caractériser de championne des employés. En effet, elle affirme la nécessité de rendre le cadre de travail le plus agréable possible étant donné la majeure partie du temps qu'y passe le travailleur dans le courant de sa vie.

\section{Analyse critique}

\subsection{Limites de la théorie}

Nous allons dans cette rubrique premièrement, discuter les limites du modèle théorique classifiant la fonction RH en quatre rôles, et deuxièmement tenter de cerner au mieux les tensions et arbitrages, les conséquences ou encore les inquiétudes que pourrait engendrer ce modèle théorique. Nous nous efforcerons de rester critique et surtout objectif par rapport à la théorie. Nous ne remettons en aucun cas le modèle en question : il est un fait certain, les quatre rôles du modèle d'Ulrich ont révolutionné le secteur des ressources humaines des entreprises.

Le champion des salariés, l'expert administratif, le partenaire stratégique ou encore l'agent de changement, tels sont les quatre rôles ou fonctions que doivent remplir au quotidien nos DRH dans nos organisations. Néanmoins une première question se pose dans nos têtes si on se met dans la peau d'un DRH : quelle fonction devrions-nous privilégier ? Le modèle d'Ulrich nous donne une vision idéale du développement de la fonction RH dans une entreprise. Il donne au secteur des ressources humaines un statut d'acteur stratégique et opérationnel capable de générer une réelle valeur ajoutée pour l'entreprise. Cependant, il ne nous communique pas un ordre d'importance au sein de ces quatre fonctions RH. Nous pensons que ce modèle se contente d'énoncer ces quatre rôles qu'il justifie alors comme des compétences essentielles pour mener à bien la fonction RH. Nous ne remettons pas en cause l'efficacité du modèle mais nous voudrions souligner le fait qu'il ne semble à priori pas évident à mettre sur pied. De plus, prises séparément, ces fonctions colleront difficilement avec l'objectif stratégique de l'entreprise.
En l'absence de toute hiérarchie au sein de ces quatre rôles, Ulrich nous laisse penser qu'un bon DRH ne sera bon que s'il se montre capable de maîtriser ces quatre fonctions. Dans la pratique, un tel DRH est très difficile à trouver. 
Une première limite que nous avons identifiée est que ces quatre rôles sont interdépendants : en clair, le modèle est performant si ces quatre fonctions sont réunies et maîtrisées. Cela peut constituer un véritable obstacle dans la mesure où la fonction partenaire stratégique, par exemple, dépasse nettement la simple gestion des ressources humaines. Nous discuterons justement cet aspect-là dans la deuxième section. 

Une autre dimension, pas fondamentale mais qui certes ne doit pas être sous-estimée, le modèle d'Ulrich date de 1993. Le contexte dans lequel ce modèle a été conçu peut s'avérer fort différent du contexte économique actuel. En effet, le secteur de l'entreprise s'est considérablement développé et c'est une bonne chose en soit. Toutefois, le monde actuel est en pleine période de récession économique, les entreprises sont sans cesse exposées à des restrictions budgétaires, licenciements de personnel et doivent sans arrêt s'adapter aux nouvelles technologies montantes. Ainsi, nous soutenons l'idée que le cadre de développement dans lequel la RH se trouve aujourd'hui semble plus instable qu'auparavant. Nous voudrions simplement attirer l'attention sur le fait que le secteur des ressources humaines est fortement concerné par cette période de récession, de licenciement et que ce contexte d'incertitude ne simplifie en rien la tâche des DRH.

\subsection{Tensions et arbitrages}

Dans cette section, il nous a été demandé de traiter des tensions et arbitrages que les directeurs de ressources humaines peuvent rencontrer au travail lorsqu'ils s'emploient à exécuter ces quatre rôles. Nous énoncerons en premier lieu les tensions à travers chacune de ces fonctions et clôturerons par illustrer quelques exemples d'arbitrages, des exemples qui font de la fonction RH un défi de tous les jours.

Tout d'abord, le rôle « champion des employés ». Ce premier rôle reprend la dimension du personnel au sein de l'entreprise. S'assurer que les travailleurs font ce qu'on leur demande tout en y trouvant une certaine satisfaction, un certain bien-être n'est guère une chose facile. Au sein d'une entreprise, chacun des membres du personnel pense, agit, réagit de manière différente. C'est un constat évident mais très important : il sous-entend tout simplement que la vie en entreprise peut parfois s'avérer conflictuelle voir problématique entre travailleurs. Un bon DRH doit pouvoir comprendre et être à l'écoute de son personnel. Cependant, motiver, écouter, encourager les travailleurs représentent un investissement important, nécessaire en tous les cas mais surtout une tâche très difficile à exécuter. Réduire les tensions internes pour permettre une meilleure entente collective au sein de l'entreprise constitue pour nos DRH un défi majeur. Ces derniers devront tout de même s'efforcer de coller au mieux aux objectifs premiers de l'organisation : gagner la motivation, la confiance des travailleurs est une bonne chose si et seulement si cette action s'inscrit dans une démarche positive et bénéfique pour l'organisation. Ce dernier argument insiste sur le côté arbitraire de cette première fonction. Le directeur des ressources humaines ne doit pas basculer et perdre de vue les objectifs stratégiques de l'entreprise.

Pour ce qui est du deuxième rôle « agent de changement », nous nous situons dans la dimension stratégique de l'entreprise du point de vue des individus. Ce rôle met en avant la capacité d'un DRH à faire face au changement susceptible d'affecter l'entreprise positivement ou négativement. Dans la pratique, cette fonction est difficile à mettre en place car nous vivons dans un monde qui évolue constamment. Les entreprises font face à de plus en plus de nouvelles technologies menaçant très fortement leur champ d'application si elles ne s'adaptent pas. 
Nous aimerions attirer l'attention sur les tensions que peut représenter cette fonction d'agent de changement. Les DRH sont soumis à une pression constante et doivent pouvoir rapidement apporter des solutions en termes d'adaptation au changement, introduire de nouvelles manières efficaces de travail et maîtriser la technologie nouvelle. En bref, un bon DRH doit faire preuve d'une grande agilité dans son adaptation continue au milieu professionnel dans lequel il évolue. Une fois confronté au changement, le DRH doit rester critique et correctement évaluer le besoin ou non d'adaptation de l'entreprise.

Le troisième rôle, c'est « l'expert administratif ». Celui-ci est essentiel au bon fonctionnement d'une entreprise. En effet, il assure que les processus, systèmes et outils RH s'emploient correctement et concordent avec  l'objectif premier de l'entreprise. Le côté expertise de cette fonction peut malmener et réellement mettre sous pression les directeurs des ressources humaines. En effet, bon nombre d'entreprises placent la barre très haut en termes d'objectifs stratégiques. L'obsession de parvenir aux objectifs fixés en fin de semestre par exemple risque de contraindre les DRH à prendre des décisions difficiles et exiger beaucoup de ces collaborateurs. Il faudra  donc rester vigilent et ne pas pousser à l'extrême cette fonction du RH. Le travailleur est avant tout un être humain et non une machine robotisée. Il ne s'agit pas de lui mettre une pression constante en plus de son travail. Un bon DRH devrait toujours prévoir une certaine marge de manœuvre au sein des tâches opérationnelles confiées aux employés. L'idéal d'un DRH serait bien entendu de trouver un juste milieu entre le champion des employés et l'expert administratif. Disposer d'un personnel qualifié, motivé et productif est un réel atout pour une entreprise. Dans la pratique, les employés se plaignent encore beaucoup de leur travail. On entend beaucoup parler de la médecine du travail. Nous n'allons pas l'aborder dans le cadre de ce travail mais cette médecine du travail de plus en plus présente dans nos organisations montre bien en quoi ce troisième rôle doit être surveillé de très près.

Le quatrième rôle s'intitule « Partenaire stratégique ». Dans sa définition, il vise à répondre aux besoins de tous les membres de l'entreprise et veille à la concordance entre la fonction RH et les objectifs stratégiques de l'entreprise. Dans la pratique, il s'agit du rôle probablement le moins bien maîtrisé par nos DRH. Il s'agit ici de proposer une réflexion quant aux moyens, aux structures et procédures que le département des ressources humaines devrait mobiliser pour aider l'organisation elle-même à atteindre ses objectifs fixés. Ce type de réflexion \footnote{\textit{Modèle d'Ulrich: qu'est-ce qui fait défaut aux RH?}, site web}, que suscite cette dernière fonction du modèle d'Ulrich, surpasse nettement les trois autres en termes de compétences. Il ne s'agit plus de la sphère individuelle mais bien collective de l'organisation. Patrons, dirigeants des ressources humaines doivent se réunir, se concerter et discuter de ces structures et procédures à mettre en œuvre. La question du budget accordé au département des ressources humaines est différente selon les entreprises et nous aimerions accorder de l'importance au fait que ce type de démarche peut s'avérer très couteux aux yeux des dirigeants d'entreprise. La part de budget accordée à la fonction RH constitue un obstacle réel à cette fonction de partenaire stratégique. 
Enfin, être un partenaire stratégique n'est pas simple et exige parfois de maîtriser de nouvelles compétences comme le marketing, la communication et faire preuve d'aisance à l'égard de l'arrivée de nouvelles technologies. En clair, il faut se montrer très talentueux pour mener à bien cette ultime fonction. Cette dernière suppose une certaine polyvalence des DRH, une qualité rare à l'heure actuelle. 



À présent nous allons en quelques lignes justifier l'aspect d'arbitrage de la fonction RH à travers le modèle d'Ulrich. La question qui se pose résulte dans la problématique de trouver un équilibre cohérent au sein de ces quatre rôles. Il nous semble évident de penser que ces quatre rôles ne seront que très rarement pleinement satisfaits. Néanmoins, il s'agit de trouver un bon compromis entre eux. À titre d'exemple, nous l'avons montré dans notre section parallèle avec le modèle théorique, nos invités DRH ne prétendent pas tout connaître et remplir parfaitement ces quatre tâches. Ils s'efforcent de trouver un équilibre soutenable et profitable pour l'entreprise au sein de laquelle ils travaillent. En tant que DRH, il n'est pas question d'être juste un champion des salariés, juste un agent de changement, un expert administratif ou encore un partenaire stratégique de l'entreprise. Ces quatre rôles seront toujours un idéal que les DRH chercheront à atteindre, ils définissent en clair les rhétoriques managériales du département des ressources humaines. 

En guise de conclusion de cette analyse, nous aimerions reprendre le point de vue d'un expert des ressources humaines, Stéphane Cadron. Nous pensons que son expérience pourrait enrichir notre travail de réflexion portant sur la fonction RH et ses implications managériales. Il préside au poste de directeur des ressources humaines chez T-groep. À titre d'information, T-groep est un organisme prestataire de services RH actif en Belgique. Stéphane Cadron nous explique ici :
\textit{Pour moi, HR continue à jouer un rôle stratégique et opérationnel. Le défi est de trouver un équilibre entre les différents rôles. Une vision claire en matière de business et de HR, assortie d'accords bien définis, contribue à ce que le processus et donc les résultats de l'entreprise aillent dans la bonne direction. Le modèle d'Ulrich est toujours à recommander pour la pratique HR : plus HR soutient toutes les facettes du business, plus la valeur ajoutée de HR augmente}\footnote{\textit{Les quatre rôles d'Ulrich restent un modèle utile pour la pratique HR}, site web}

\section{Proposition pour l'avenir}

Après avoir procédé à l'analyse critique et l'identification des limites du modèle théorique, nous allons dans cette ultime section proposer un diagnostic sur les compétences attendues pour les directeurs RH de demain. Une première partie de réflexion sur le cadre économique et l'évolution du rapport de l'Homme avec son milieu. Une deuxième partie où on tente de lister les rôles ou fonctions que nous jugeons primordiales pour la réussite de nos futurs DRH.

Tout d'abord, le point de vue économique. Nous avons traversé de nombreuses crises économiques importantes. Ces crises deviennent toujours plus récurrentes et démontrent un profond mal-être de notre économie. Cela représente un fameux challenge pour nos DRH de demain qui se devront de penser à un nouveau système plus centré sur l'Homme. De plus, nous voyons une certaine égalité de plus en plus présente dans notre société entre les différents sexes, cultures, etc. En quelques mots, on constate une volonté grandissante de donner du sens à nos actions. Au niveau des modes d'organisation du travail, ceux-ci ont bien évolué avec la disparition des frontières et une communication ininterrompue due à la globalisation. Ceci nous amène à penser que l'humanité entre dans une nouvelle ère.

D'un point de vue plus général, Hartmut Rosa \footnote{ROSA, H., (2005)} a développé une théorie sociale que nous jugeons pertinente pour caractériser cette nouvelle ère. Rosa observe d'abord un phénomène nouveau de compression du présent. Il le définit comme \textit{la diminution générale de la durée pendant laquelle règne une sécurité des attentes concernant la stabilité des conditions de l'action}. Cette théorie s'articule autour d'un concept qu'il nomme accélération. Rosa distingue trois formes d'accélération : l'\textit{accélération technique}, l'\textit{accélération du changement social} et l'\textit{accélération du rythme de vie}. L'accélération technique a changé notre rapport au temps et à l'espace. La technologie est devenue un outil de travail incontournable. L'accélération du changement social plonge les individus dans  un sentiment partagé d'incertitude. L'accélération du rythme de vie est la résultante logique des deux autres. De fait, nous assistons à un paradoxe de l'Homme pressé : le progrès technologique a permis à l'individu de gagner du temps mais d'autre part, ce dernier se plaint constamment de manquer de temps. Pour nous résumer, qui dit nouvelle ère dit nouveaux enjeux économiques, politiques et environnementaux auxquels les ressources humaines se devront de faire face.
Nous allons maintenant discuter le degré d'importance accordé à chacune des quatre fonctions suites à ces changements.

Pour commencer, le rôle d'agent de changement prend une dimension plus importante qu'autrefois. En effet, cette instabilité implique un besoin d'adaptation permanent. Nous sommes de ce fait amenés à penser que la structure de l'entreprise d'aujourd'hui, en particulier des RH, ne suffit plus pour suivre ce flot de changements. Si cette fonction est amenée à prendre davantage d'ampleur, à force de travail égale, les trois autres fonctions seront inévitablement délaissées. De fait, la performance économique étant le mode de survie de toute entreprise, il ne sera pas possible que celle-ci n'accorde pas plus de moyens au rôle d'agent de changement. D'une certaine manière, si les ressources consacrées à la fonction RH ne sont pas augmentées, il faudra en transférer une partie des autres fonctions.

Une facette supplémentaire de la problématique de changement réside dans un accroissement des responsabilités sociales des entreprises vis-à-vis de la société. Cette responsabilité sociale nous pousse à soutenir l'idée que la fonction de champion des employés au sein des entreprises ne peut en aucun cas être négligée. Il en découle que si l'entreprise ne veille pas au bien être de ses travailleurs, elle risque une dégradation de son image ou pire, une dégradation de sa performance économique. Il n'est désormais pas dans l'intérêt des organisations de sous-estimer le pouvoir d'action de cette fonction RH. Néanmoins, l'entreprise reste un lieu de travail et en aucun cas un centre de loisirs : les travailleurs doivent être respectés et pouvoir travailler dans de bonnes conditions mais le bien-être se limite à cela. La fonction de champion des salariés a bien évolué puisque les conditions de travail ont fortement été améliorées. Néanmoins, il y a toujours des exceptions. À titre informatif, on pense notamment à la vague de suicides qui a touché l'entreprise France Telecom\footnote{DELOBBE,N., Cours de management humain, séances 1 et 2}.

En raison des avancées technologiques, on peut penser que la fonction d'expert administratif ne représentera plus nécessairement un défi majeur de la fonction RH. De nombreux outils informatiques s'offrent désormais aux entreprises leur permettant d'automatiser la plupart des procédures, processus ou procédés de travail. La communication au sein même de l'entreprise est facilitée par la technologie. Pour dire simple, le progrès technologique contribue fortement à la fonction RH. Dans le cas de l'expert administratif, une partie du travail est relayée par les systèmes informatiques.

En dernier, nous nous penchons sur la fonction du partenaire stratégique. Cette dernière prime sur les trois autres mais reste la plus difficile à mettre en œuvre. Comme nous l'avons abordé dans la section précédente, elle dépasse largement la sphère individuelle et s'oriente vers la sphère collective. Managers, DRH, patrons des entreprises doivent impérativement procéder à des réunions régulières, discuter la question budgétaire et au besoin revoir les objectifs stratégiques de l'entreprise. 

Pour conclure, nous tenons à rappeler qu'un bon DRH devra se résoudre à trouver une forme d'équilibre au sein de ces quatre fonctions. Nous soutenons l'idée simple que le profil du DRH de demain s'apparente à un acteur ou agent de changement, un champion des employés et un partenaire stratégique de l'entreprise. Tels sont les compétences requises pour mener à bien cette mission. 

\section{Les limites de notre approche}

Dans notre travail, nous n'avons pas envisagé le double tranchant du progrès technologique. D'un côté, la technologie peut s'avérer très bénéfique pour l'entreprise en simplifiant le travail et en augmentant son efficacité. De l'autre côté, elle peut être néfaste pour les individus qui y sont constamment exposés. Avec l'avancée majeure du télétravail, de l'e-learning, l'employé doit être constamment connecté à son périphérique multimédia, ce n'est plus un accessoire mais bien un besoin, une obligation, voire même une dépendance. L'Homme pourrait tendre à s'isoler dans sa sphère sociale. À cela s'ajoute, de manière plus générale, la pression que subit le travailleur à vouloir rester performant et donc à devoir s'adapter de manière la plus rapide possible aux nouveaux changements technologiques et voire même à travailler plus qu'il n'est prévu,  cela, de peur de se faire remplacer. À l'extrême, mais peut-être dans un futur proche, l'homme ne vivrait que pour son travail et n'appréhenderai pas une machine mais en deviendrait une. 
Une proposition de rôle, voire même une mission innovatrice pour un DRH serait de mettre l'accent sur l'aspect interactif, collectif des relations de travail au sein des membres du personnel dans le but de créer un réel " team spirit ". Ceci pourrait bel et bien fonctionner avec des expériences de " team building ". En s'inspirant du management suédois présent chez Ikea, on pourrait penser à instaurer des pauses collectives ou encore le café collectif avant d'entamer une journée de travail. Le cas d'Ikea suscite une piste de réflexion dans le sens où il propose une forme de management nettement plus humaine essentiellement basée sur la fonction de champion des salariés. On pourrait s'interroger sur son efficacité et le comparer aux autres formes actuelles de management présent dans nos organisations.   

\section{Bibliographie}

\noindent CONNER,J. et ULRICH,D.,(1996),"Human resource roles: creating value, not rhetoric", dans HR. \textit{Human resource planning,} vol.19,\no 3, pp 38-49
\vspace{0.5cm}

\noindent BOURGINE, B., (2014), "Les religions dans la modernité européenne", Syllabus du cours de Questions de sciences religieuses: christiannisme et questions de sens, Université catholique de Louvain, Faculté des Sciences Economiques, Sociales et Politiques, Chapitre 1.

\vspace{0.5cm}

\noindent ROSA, H., (2005), "Accélération. Une critique sociale du temps" trad. D. RENAULT, Coll. Théorie critique, Paris, La Découverte, 2010,pp 35-143

\vspace{0.5cm}

\noindent http://www.learninglive.be/fr/modele-dulrich-quest-ce-qui-fait-defaut-aux-rh, \textit{Modèle d'Ulrich: qu'est-ce qui fait défaut aux RH?}, mise à jour le 07/05/2013, consulté le 31 mars 2015.

\vspace{0.5cm}

\noindent http://www.ascento.be/fr/nouvelles/detail/les-quatre-r\%C3\%B4les-d-ulrich-restent-un-mod\%C3\%A8le-utile-pour-la-pratique-hr , \textit{Les quatre rôles d'Ulrich restent un modèle utile pour la pratique HR},  dernière mise à jour le 19/06/2013, consulté le 31 mars 2015.

\vspace{0.5cm}

\noindent DELOBBE, N., (2005-2006), cours de management humain, Université catholique de Louvain, Faculté des Sciences Economiques, Sociales et Politiques.


\end{document}